%
\documentclass{book}

\usepackage[utf8]{inputenc}
%\usepackage[spanish]{babel}
%\usepackage[spanish]{isodate}
\usepackage[final]{listings}
\usepackage{courier}
\usepackage{multirow}
\usepackage{enumitem}
% For links
\usepackage[bookmarks]{hyperref}
\hypersetup{colorlinks=true,allcolors=blue}

\usepackage[pdftex]{graphicx}
\usepackage[toc,page]{appendix}
\usepackage[nottoc]{tocbibind}

\usepackage{longtable}
\usepackage{pbox}
%\usepackage[sorting=none]{biblatex}
\usepackage{gensymb}
\usepackage{epigraph}
	
\usepackage{todonotes}

\usepackage{titling}
\usepackage{setspace}


\def\arraystretch{1.1}

%\renewcommand{\chaptername}{Capítulo}
%\renewcommand{\figurename}{Figure}

\lstset{	
	basicstyle=\footnotesize\ttfamily,
	language=bash,
	breaklines=true,
	fontadjust=true
}

\bibliographystyle{abbrv}
%\bibliographystyle{apalike}

\hyphenation{detection camera cameras Systems}

\begin{document}

\begin{titlepage}
    \begin{center}
        \vspace*{1cm}
        \Huge
        
        Multisensor Architecture for an Intersection Management System
    
        \vspace{4.5cm}
        \LARGE
        Gustavo Adolfo Velasco-Hernandez
        
        \vfill
        \large
        \vspace{0.8cm}
        
        %\includegraphics[width=0.4\textwidth]{university}
        
        School of Electrical and Electronics Engineering\\
        Universidad del Valle\\
        Colombia\\
        \today
        
    \end{center}
\end{titlepage}
\begin{titlepage}
    \begin{center}
        \vspace*{1cm}
        \Huge        
        Multisensor Architecture for an Intersection Management System
   
        \vspace{3.5cm}
        \LARGE
        Gustavo Adolfo Velasco-Hernandez
        
        \vfill
        \large
		        
        
        A thesis presented in partial fulfillment of the Requirement for the degree of Master of Science in Engineering with emphasis on Electronics.
        
        \vspace{2cm}
        
        Advisor: Prof. Eduardo Caicedo Bravo
		
		\vspace{2cm}
        
        %\includegraphics[width=0.4\textwidth]{university}
        
        School of Electrical and Electronics Engineering\\
        Universidad del Valle\\
        Colombia\\
        \today
        
    \end{center}
\end{titlepage}

%\title{Multisensor Architecture for an Intersection Management System\vspace{8cm}}
%\vfill
%\vfill
%\vfill
%\author{Gustavo Velasco-Hernandez\vspace{5cm}}
%\date{\today}
%\maketitle
%
%\title{Multisensor Architecture for an Intersection Management System}
%\vfill
%\vfill
%\vfill
%\author{Gustavo Velasco-Hernandez}
%\vfill
%\vfill
%\vfill
%\date{\today}
%\maketitle

\chapter*{Abstract}
\addcontentsline{toc}{chapter}{Abstract}%
Intersections or juntions are critical points in transportation systems due to their dynamic nature, making them prone to safety or efficiency problems. For this reason the use of technology in monitoring intersections have been increasing over the last years. After a comprehensive review on this topic, a  multisensor architecture for an intersection management system is conceived, having as main features its scalability and modularity. This architecture is composed by 4 main elements: communication structure, data model, information structure and processing blocks. An example implementation of the architecture is made using the publisher/subscriber approach along with different tools and technologies that allow the whole system to fulfill the proposed criteria, to be scalable and modular. After testing the system using a dataset containing data from camera sensors and lasers sensors, some comments about its results, future developments and integration with current needs and new trends, are discussed.


%\chapter*{Acknowledgements}%
%\addcontentsline{toc}{chapter}{Acknowledgements}%

%I would like to thank to all people that in some way or another BLABLABLA


\tableofcontents
\setcounter{tocdepth}{3}

\listoffigures

\listoftables

% Introduction
\chapter [Introduction]{Introduction}
\chaptermark{Introduction}

\section{Goals}

The main goal for this proposal is to conceive a multisensor architecture for an intersection management system and deploy it in a simulated scenario. In order to achieve this goal, these objectives were set:

\begin{itemize}
\item To make a literature review on intelligent transportation systems, focusing on developments about intersection management
\item To define system architecture and specifications, and choose a platform which fits requirements for the deployment
\item To develop and implement an intersection monitoring system based on laser and video sensors using sensor fusion techniques
\item To propose and deploy a method for incident estimation and warning generation
\item To set a test and evaluation protocol for the developments made on the choosen platform 
\end{itemize}

\section{Scope}


%\begin{document}

\chapter [Multisensor Data Fusion and Intersection Management Systems]{Multisensor Data Fusion and Intersection Management Systems}
\chaptermark{Sensor Fusion and IMS}

An intersection is a highly-dynamic scenario that can be monitored using a wide range of sensors. For this reason, an efficient and accurate fusion of the information is needed. This chapter is divided into two sections. In the first section a brief overview of multisensor data fusion is presented, remarking in different architectures proposed in the literature and different algorithms and frameworks used for this task. The second section contains a short review on intelligent transportation systems and intersection management systems, including a description of elements involved in the development of an IMS application. Finally, some projects that include multisensor data for intersection managing applications are presented.

\section{Multisensor Data Fusion}

Data fusion, also referred as mutisensor data fusion, information fusion or sensor fusion, has received several definitions from different authors in the literature. For example, Joint Directors of Laboratories defined data fusion as "multi-level, multifaceted process handling the automatic detection, association, correlation, estimation, and combination of data and information from several sources"\cite{White1991}. Luo refers to multisensor fusion and integration as "synergistic combination of sensory data from multiple sensors to provide more reliable and accurate information"\cite{Luo2002} and "to achieve inferences that are not feasible from each individual sensor operating separately"\cite{Luo2011}. Elmenreich states that sensor fusion is "the combining of sensory data or data derived from sensory data such that the resulting information is in some sense better than would be possible when these sources were used individually"\cite{Elmenreich2007}. In \cite{Bostrom2007} there is a compilation of more definitions of information fusion and the author summarize in his own statement as follows: "Information fusion is the study of efficient methods for automatically or semi-automatically transforming information from different sources and different points in time into a representation that provides effective support for human or automated decision making".  

% Elmenreich in [Elmenreich07]: Sensor fusion as "the combining of sensory data or data derived from sensory data such that the resulting information is in some sense better than would be possible when these sources were used individually"

%Khaleghi in [Kaleghi11]: "Multisensor data fusion is a technology to enable combining information from several sources in order to form a unified picture"

%White, JDL in [White91]: Data fusion as "multi- level, multifaceted process handling the automatic detection, association, correlation, estimation, and combination of data and information from several sources"

%Bostrom in [Bostrom07]: "Information fusion is the study of efficient methods for automatically or semi-automatically transforming information from different sources and different points in time into a representation that provides effective support for human or automated decision making"

%Luo refers to multisensor fusion and integration as "synergistic combination of sensory data from multiple sensors to provide more reliable and accurate information"\cite{Luo2002} and "to achieve inferences that are not feasible from each individual sensor operating separately"

All of previous definitions can be seen as a way to answer these three questions about data fusion:
\begin{itemize}
	\item {What is involved in data fusion?\\Combine, merge or integrate homogeneous or heterogeneous data.}
	\item {What is the aim of data fusion?\\Get a better representation of a process or the environment, infer underlying information, improve quality of the data.}
	\item {How to apply data fusion?\\Data fusion is a multi-level task, depending of the nature of the sensors, the final application and the environment.}
\end{itemize}

It is clear, now, that multisensor data fusion is a multidisciplinary field, because information in a typical process, flows from sensors to applications, passing through stages of filtering, data enhancement and data extraction. It is for this that knowledge in a wide range of fields are required, e.g. signal processing, machine learning, probability and statistics, etc. Also, it would be pointless to try to define a general method, technique or architecture that fits the requirements of any system, for applying data fusion in it.


%\subsection{Overview}
\subsection{Data Fusion Architectures and Models}

Although there is not a general rule of how to design or implement a sensor fusion system, many authors have proposed some models, architectures and guidelines for this task. Three well-known models are Waterfall model, JDL fusion model and Multisensor integration fusion model.

\subsubsection{Waterfall model}

Harris and Markin in \cite{Harris1998} and CITE, proposed a model named Waterfall, in which they describe the fusion process as an information flow through sensing to decision-making. They describe 3 levels of processing with 2 inner stages each ((\ref{WaterfallModel}). The first level is about transform the raw data from sensor to a better representation of the measured phenomena through signal processing and having in mind sensor models and nature of the process itself. The second level objective is to find a meaningful description of the data, reducing its volume while maximising information. This is done using feature extraction and pattern recognition techniques. The third level is the high level of the process in which situation assessment and decision making are performed, based on data available, configuration parameters, database information or human interaction. Finally, a feedback from high-level to low-level (sensor) is done, advising the whole system for re-calibration or reconfiguration.

\begin{figure}[ht!]
\centering
\includegraphics[scale=0.4]{fig/2/WaterfallModel.png}
\caption{Waterfall model (from \cite{Esteban2005}).}
\label{WaterfallModel}
\end{figure}

\subsubsection{JDL fusion model}

One of the first proposals of fusion architecture, and probably one of the most widely used, is the JDL fusion model, originated from the US Joint Directors of Laboratories and described by Hall and Llinas in \cite{Hall1997} and \cite{Llinas1998}. The JDL fusion model was conceived to aid the developments of military applications and comprises 5 levels of data processing at which data fusion could be done. These levels and a database are connected by a bus (\ref{JDLmodel}), and are not meant to be execute sequentially and can also be executed concurrently. 

\begin{figure}[ht!]
\centering
\includegraphics[scale=0.4]{fig/2/jdlmodel.png}
\caption{JDL fusion model. (from \cite{Hall1997}).}
\label{JDLmodel}
\end{figure}

The first stage, referred as level-0 is the source preprocessing in which raw data is handled to concentrate the more pertinent data for the current situation. The level-1 is for object refinement, starting with the alignment of the data in a commonly space-time reference frame. Then, performs identification and tracking of objects using different techniques. Situation refinement is at level-2, which takes observed and partially-observed object from level-1 and tries to find a contextual description between them. Level-3, threat refinement, is the level in which results from level-2 are interpreted looking for possible advantages and disadvantages for the system to operate, based on previous knowledge and predictions about executing an action.

\subsubsection{Multisensor Fusion Integration model}

Luo and Kai, in \cite{Luo1989, Luo1990}, proposed a full integration model for data fusion in which they define a three-level hierarchy for sensor fusion: data-fusion, feature-fusion and decision-fusion. This model separate MFI in five classes, based on Input/Output pair: Data in-data out fusion, data in-feature out fusion, feature in-feature out fusion, feature in-decision out fusion, and decision in-decision out fusion \cite{Luo2011} (figure \ref{fusionClasses}).

\begin{figure}[ht!]
\centering
\includegraphics[scale=0.4]{fig/2/fusionClasses.png}
\caption{Five classes of Multisensor Fusion.}
\label{fusionClasses}
\end{figure}

Also, they made a clear distinction between multisensor fusion and multisensor integration, being the former the process in which information provided by a set of sensors is combined in any of the three levels aforementioned, and the latter is how sensor fusion could be integrated in a full system in order to assist in a particular task. As is depicted in figure \ref{MFIArch}, sensor fusion is an element of the whole MFI architecture, which also includes block for sensor managing tasks, like control, selection and registration of sensors, previously to the fusion process. A sensor processing stage and a system controlling module are also included after the sensor fusion stage.

\begin{figure}[ht!]
\centering
\includegraphics[scale=0.4]{fig/2/MFIArch.png}
\caption{Multisensor Fusion Integration architecture (from \cite{Luo2011}).}
\label{MFIArch}
\end{figure}

\subsection{Classification of Data Fusion Architectures}

Elmenreich in \cite{Elmenreich2007} classify fusion models in three categories: Abstract models, generic architectures and rigid architectures. Abstract model are not intended to show how to implement a sensor fusion system, but to explain which processes are done in it. Generic architectures gives an outline on how a system could be implemented in an application, but do not specify what type of hardware, database or communication system could be used. Finally, rigid architectures, are a good guide for implementation  of data fusion in certain applications, at the cost that several design decisions have been already taken, making expensive the migration to another architecture. 

In addition to models previously mentioned, there exists more proposal for data fusion architectures in literature, as can be viewed in table \ref{fusionModelsClas}.

% TODO: Change for a table
%\begin{figure}[ht!]
%\centering
%\includegraphics[scale=0.6]{fig/2/fusionModelsClas.png}
%\caption{Classification of some data fusion models}
%\label{fusionModelsClas}
%\end{figure}


\begin{table}
\footnotesize
\centering
\begin{tabular}{|c | c|}
\hline
\textbf{Category} & \textbf{Data Fusion Model} \\
\hline
\multirow{2}{*}{Abstract Model} & Waterfall Model \cite{Harris1998} \\
& Boyd Loop \cite{Boyd1987} \\
\hline
\multirow{3}{*}{Generic Architecture} & JDL Model \cite{White1991} \\
& Multisensor Fusion Integration Model \cite{Luo1989} \\
& Omnibus model \cite{Bedworth2000} \\
\hline
\multirow{3}{*}{Rigid Architecture} & LAAS Architecture \cite{Alami1998} \\
& DFuse Architecture \cite{Kumar2003} \\
& Time-triggered Model \cite{Elmenreich2001} \\
\hline
\end{tabular}
\caption{Classification of data fusion models}
\label{fusionModelsClas}
\end{table}


\subsection{Algorithms in Data Fusion}

Different types of algorithms have been used in implementing data fusion systems, depending on a variety of conditions like, level of fusion, type of the data, nature of environment, etc. Constrains like processing and memory limitations, centralised or distributed schemes, human-interactive or completely autonomous process, also determine which algorithms should be used in fusion process.

Luo in \cite{Luo2011} propose a classification for fusion algorithms based on the level of fusion. Low-level fusion refers to the merge of raw data or signals, mid-level fusion refers to the fusion of features and High-level fusion refers to the process of fuse decisions. Khaleghi in \cite{Khaleghi2013} describes a classification for fusion algorithms based on challenging problems that arise from the data to be fused, due to the variety of sensors and the nature of the application environment. This classification is not on the algorithms directly, but on the theory or framework in which they originate. Four types of data are enumerated: Imperfect data, correlated data, inconsistent data and disparate data. These two approaches of fusion algorithms classification are summarized in tables \ref{fusionAlgClasLuo} and \ref{fusionAlgClasKhaleghi}.

\begin{table}[ht!]
\footnotesize
\centering
\begin{tabular}{|c|c|c|c|}
\hline
\multicolumn{2}{|c|}{\textbf{Low level fusion}} & \textbf{Medium level fusion} & \textbf{High level fusion} \\
\hline
\multicolumn{2}{|c|}{\textit{Estimation methods}} & \textit{Classification methods} & \textit{Inference methods} \\
\hline
\parbox{2.5cm}{
	Recursive:
	\begin{itemize}[leftmargin=.07in]
		\item Kalman filter
		\item Extended Kalman filter
	\end{itemize}
	Non-Recursive:
	\begin{itemize}[leftmargin=.07in]
		\item Weighted average
		\item Least squares
	\end{itemize}
	}
 & 
 \parbox{2.5cm}{	 
 	Covariance-based:
	\begin{itemize}[leftmargin=.07in]
		\item Cross covariance
		\item Covariance intersection
		\item Covariance union
	\end{itemize}
	}
&
 \parbox{3cm}{	 
	\begin{itemize}[leftmargin=.07in, noitemsep]
		\item Parametric templates
		\item Cluster analysis
		\item K-means clustering
		\item Learning vector quantization
		\item Kohonen feature map
		\item Artificial neural networks
		\item Support vector machines
	\end{itemize}
	}
&
\parbox{3cm}{	 
	\begin{itemize}[leftmargin=.07in, noitemsep]
		\item Bayesian inference
		\item Particle filters
		\item Dempster-Shafer theory
		\item Expert systems
		\item Fuzzy logic
	\end{itemize}
	} \\
\hline
\end{tabular}
\caption{Classification of fusion algorithms based on level of fusion}
\label{fusionAlgClasLuo}
\end{table}

\begin{table}[ht!]
\footnotesize
\centering
\begin{tabular}{|c|c|}
\hline
\textbf{Data Problem} & \textbf{Framework} \\
\hline
\multirow{7}{*}{Imperfection} & Probabilistic \\
& Evidential \\
& Fuzzy reasoning \\
& Possibilistic \\
& Rough set theoretic \\
& Hybridization \\
& Random set theoretic \\
\hline
\multirow{2}{*}{Correlation} & Correlation elimination \\
& Correlation presence \\
\hline
\multirow{6}{*}{Inconsistency} & Sensor validation \\
& Stochastic sensor modeling\\
& Prediction \\
& Augmented state framework \\
& Combination rules \\
& Dempsters' rule \\
\hline
\multirow{3}{*}{Disparateness} & Dempster-Shafer theoretic framework \\
& Human-centered data fusion \\
& Hard-soft data fusion \\
\hline
\end{tabular}
\caption{Classification of fusion algorithms based on challeging problems in data}
\label{fusionAlgClasKhaleghi}
\end{table}

%\subsection{Laser and Video Data Fusion}

\section{Intersection Management Systems}

Intelligent Transportation Systems includes a wide range of applications and services transversal to many knowledge areas. For classifying those services, some taxonomies have been proposed like the ones presented in \cite[Ch.1]{Sussman2005} and \cite{Williams2008}. From described categories and classes, Advanced Traffic Management Systems have to be considered when an intelligent handling of traffic needs to be deployed.

One of the most desirable scenarios to improve efficiency and safety is an intersection. This because intersections are places where vehicles arrive from different directions at different velocities, increasing the chances for incidents and crashes. Choi \cite{Choi2010} states that 40\% of reported traffic accidents in the US, were intersection related. Also, in  \cite{CorporacionFondodePrevencionVial2010}, is reported that for Colombia in 2011, most of the accidents in the main cities were at intersections.

%El concepto de sistemas inteligentes de transporte (ITS), presenta un amplio campo de acci\'{o}n transversal a diferentes \'{a}reas del conocimiento e igualmente presenta una gran variedad de aplicaciones y servicios. Para la clasificación de estos servicios se han definido diferentes taxonom\'{i}as como las descritas en \cite[Ch.1]{Sussman2005} y en \cite{Williams2008}. De las clases y categor\'{i}as descritas, se tiene un subgrupo de los ITS que debe ser contemplado para cumplir con el objetivo de este proyecto, que es el que incluye los servicios y sistemas para la administraci\'{o}n y operaci\'{o}n del tr\'{a}fico (o Sistemas Avanzados de Administraci\'{o}n de Tr\'{a}fico, del ingl\'{e}s ATMS, Advanced Traffic Management Systems).
%
%Uno de los escenarios donde se busca mejorar la eficiencia y la seguridad usando estos sistemas son las intersecciones viales, ya que son puntos donde se encuentran veh\'{i}culos en diferentes direcciones a diferentes velocidades, lo cual incrementa la probabilidad de incidentes y choques. Cerca del 40\% de los accidentes de tr\'{a}nsito reportados en Estados Unidos en 2008, estaban relacionados con intersecciones \cite{Choi2010}. En Colombia, para el 2011 la mayor\'{i}a de accidentes en las principales ciudades del pa\'{i}s se concentraba en intersecciones \cite{CorporacionFondodePrevencionVial2010}.

%\section[Types of IMS]{Types of Intersection Management Systems}

Different types of applications and systems are conceived to address these issues. Some tasks performed by those systems are intersection monitoring, vehicles detection, incident warning, collision avoidance, among others. A typical Intersection Management System is composed by three main components: Data source, that could be infrastructure sensors, like inductive loops, range sensors or cameras, and vehicle sensors and traveling data; decision system, which is the core of the whole system, is in charge of analyse and process information provided by infrastructure, vehicles and authorities in order to identify objects, recognise patterns, predict future incidents, control traffic and generate safe decisions and warnings alerts; and finally, is the presentation and displaying of the output of decision system, through infrastructure using dynamic signals, traffic light controlling, or using direct communication with drivers or vehicles through on-board visualisation/notification system. A block diagram of a generic IMS is presented in figure \ref{arch}

%Para afrontar esta problem\'{a}tica aparecen diferentes tipos de aplicaciones y sistemas para realizar tareas como monitoreo y administraci\'{o}n de intersecciones, detecci\'{o}n de veh\'{i}culos, advertencia de incidentes, prevenci\'{o}n de colisiones, entre otros. El esquema de comuniciaci\'{o}n de estos sistemas se basa en la interacci\'{o}n entre los veh\'{i}culos y la infraestructura, com\'{u}nmente dividida en 3 tipos: Comunicaci\'{o}n veh\'{i}culo a veh\'{i}culo (V2V), veh\'{i}culo a infraestructura (V2I) y la combinaci\'{o}n de las dos anteriores (V2X).

\begin{figure}[ht!]
\centering
\includegraphics[scale=0.55]{fig/2/genericIMS.png}
\caption{Generic block diagram of an Intersection Management System.}
\label{arch}
\end{figure}



%En la figura \ref{arch}, se presenta un diagrama de bloques general de un sistema de administraci\'{o}n de una intersecci\'{o}n en donde se destacan 3 componentes: (a) Las fuentes de los datos, que pueden ser sensores en la infraestructura, como sensores inductivos, sensores de rango, c\'{a}maras, o sensores en el veh\'{i}culo. (b) Sistema  de decisi\'{o}n o administraci\'{o}n, que es el encargado de analizar la informaci\'{o}n de los veh\'{i}culos y el entorno, predecir incidentes futuros y enviar se\~{n}ales de alerta. y (c) La presentaci\'{o}n de informaci\'{o}n de control, que puede ser en la infraestructura con se\~{n}ales din\'{a}micas, control de los sem\'{a}foros, como tambi\'{e}n puede ser en el veh\'{i}culo mediante un dispositivo de visualizaci\'{o}n.

\subsection{Components in IMS application}

Intersection monitoring is a required task to be done within intelligent transportation systems for high-level applications like traffic analysis, counting and classification of vehicles or pedestrians, event prediction, incident detection and security and surveillance systems. Those applications have to take into account some of the elements depicted in figure \ref{arch} and developments in IMS have a wide range of approaches and objectives. In order to study IMS applications, five components have been defined, which are present on these applications, and on most cases, more than one component could be involved in the same development. In figure  \ref{imsComps} a graph is presented, showing aforementioned components and elements within them, and next, a description of each component is given.

\begin{figure}[ht!]
\centering
\includegraphics[scale=0.25]{fig/2/ims_graph5.png}
\caption{Components of an Intersection Management System application.}
\label{imsComps}
\end{figure}

\subsubsection{Application}

Application component could be seen as the final objective of the system. Generally, this includes high-level tasks like monitoring, analysis or control. Monitoring or surveillance systems execute actions like recognition, detection and/or tracking of objects in the scene. 
Other systems analyse the behaviour and interactions between detected objects to recognise path patterns, determine the context of the environment and predict some events of interest. At a higher level there are systems which make decisions based on detection of certain traffic conditions to handle traffic lights, control intersection access, generate warnings to drivers or issue traffic tickets when a rule violation exists.

\subsubsection{Data Source}

The origin of data is considered an independent component because of the variety of possible sources and posterior processing stages. From infrastructure side, data can be captured using a wide range of sensors like inductive loops, lasers, lidars and cameras. Also, monitoring connections to wireless networks. On the other side, data from vehicles is also useful for the system to enhance its representation of the scene and take decisions. This could be low-level data like vehicle state variables, for example, speed, orientation, acceleration, etc., or high-level data like travel information.


\subsubsection{Target}

In an intersection many objects of different kinds interact between them. Pedestrian and vehicles are found at intersections, and latter includes bicycles, motorcycles, cars, vans, buses, trucks and some other types of vehicles. For this reason some applications are designed for a specific element or group of elements. Pedestrian tracking, motorcycles recognition or car counting are examples of targeted applications.

\subsubsection{Communication}

One of the keypoints of ITS is how information technologies and communication advances are included in transportation. The main goal of this is to allow information sharing between vehicles and infrastructure entities. For this reason, 3 communication approaches appear: Vehicle-to-vehicle or Inter-vehicle communication (V2V), vehicle-to-infrastructure communication (V2I) and vehicle to both vehicle and infrastructure (V2X). Several protocols and standars have been proposed for these communication approaches, for example DSRC, WAVE and IEEE 802.11p, but research and development on this component is still active.

\subsubsection{Implementation}

Not all IMS applications are implementable on a real scenario, maybe because is not the scope of the application or because it is in a early stage and it could be validated in other ways. This types of developments are sometimes implemented and evaluated using simulators, making functional prototypes or deploying scale-models. Some other projects use datasets to evaluate new algorithms and data processing techniques, and then compare obtained results with previous works. Augmented reality is also used as a tool for evaluating and validate developments, taking advantage of the interaction of a real system with a simulated/artificial scenario.

%many approaches have been proposed for information capture stage, both taking data from vehicle or from the infrastructure. For example, monitoring mobile wireless network like Bluetooth \cite{Friesen2013}, detecting vehicle presence using inductive loops or RFID tags, identifying vehicles and pedestrian using cameras \cite{Buch2011, Strigel2013}, single or multiple laser-scanners \cite{Meissner2010, Meissner2012, Zhao2006, Zhao2008, Zhao2012}, and multi-sensor integrated systems \cite{Meissner2013, Meissner2013a, Meissner2014, GoldHammer2012, Zhao2009}.

%Other application are traffic-control oriented, rather than to sensing and data-processing stage. The aim of these applications are to maximise vehicle flow, reduce accidents and improve safety. Many works have been done and although their objective is the same, are diverse between them regarding its communication approach (V2V, V2I or V2X), sensors and information that are used, the execution scenario (simulation, real implementation or augmented reality) and the nature of vehicle driving (Human driver or autonomous vehicles).

%El monitoreo en las intersecciones viales es una tarea requerida en los sistemas inteligentes de transporte para aplicaciones de alto nivel como  an\'{a}lisis de tr\'{a}fico, conteo y  clasificaci\'{o}n de veh\'{i}culos, detecci\'{o}n de eventos, prevenci\'{o}n de accidentes y aplicaciones de seguridad y vigilancia. Para cumplir con estas tareas se han propuesto diversas soluciones para obtener la informaci\'{o}n del entorno, ya sea proveniente de los veh\'{i}culos, por ejemplo monitoreando conexiones a redes m\'{o}viles, como Bluetooth \cite{Friesen2013}, detectando su presencia usando sensores inductivos o etiquetas RFID, y tambi\'{e}n tomando datos desde la infraestructura, usando c\'{a}maras  \cite{Buch2011}, sensores l\'{a}ser o arreglos de estos \cite{Zhao2012} o sistemas que integran m\'{u}ltiples tipos de sensores \cite{Zhao2009}\cite{Pyykonen2010}. Otras aplicaciones adem\'{a}s de monitorear y supervisar las intersecciones, est\'{a}n orientadas al control del tr\'{a}fico en estas con  el objetivo de maximizar el flujo, disminuir la accidentalidad y aumentar la seguridad. Diversos trabajos se han hecho y aunque su finalidad es similar, existe variedad entre s\'{i}, ya sea seg\'{u}n el tipo de comunicaci\'{o}n en la que se basan (V2V, V2I o V2X), el tipo de sensores e informaci\'{o}n que usan, el entorno de ejecuci\'{o}n (simulaci\'{o}n, real o realidad aumentada) y el tipo de veh\'{i}culos (con conductores humanos o veh\'{i}culos aut\'{o}nomos).

%In \cite{Ball2010} and \cite{Azimi2012}, intervehicular-communication based architectures are proposed. In these, each vehicle send to the others its own information about position, velocity, destination and more, and then they coordinate how the access to the intersection should be done. In \cite{Guerrero-Ibanez2013}, is presented a system capable of work in V2I mode, additional to V2V, using inductive loops and RFID information for detecting vehicles. The information is gathered by an intersection controller which manage the access to the crossroad. Another V2I based systems is presented in \cite{Basma2011}, where wireless magnetic nodes are used to detect presence and to send data to a base station, which determines possible collisions and warns driver through a visualisation system in infrastructure. Furthermore, in \cite{Dresner2008} and \cite{CondeBento2012}, are presented simulation-based systems for traffic controlling, using agents and applied to autonomous vehicles. In \cite{Quinlan2010}, an augmented reality implementation of the system proposed in \cite{Dresner2008} is done. In this work, approaching vehicles request to an intersection manager for authorisation to cross and is the manager which decides if it is safe to cross or not, depending on the defined traffic policies.

%En \cite{Ball2010} y \cite{Azimi2012} se proponen arquitecturas basadas en comunicaci\'{o}n intervehicular, donde cada auto env\'{i}a a los dem\'{a}s informaci\'{o}n correspondiente a su posici\'{o}n, velocidad, destino y otros datos, y as\'{i} coordinan el acceso a la intersecci\'{o}n. En \cite{Guerrero-Ibanez2013} se presenta un sistema que adem\'{a}s de trabajar en modo V2V, presenta la opción V2I que incluye sensores inductivos y m\'{o}dulos RFID para detectar veh\'{i}culos. La informaci\'{o}n adquirida es procesada por un controlador de intersecci\'{o}n, el cu\'{a}l se encarga de administrar el ingreso al cruce vial. Otro sistema basado en V2I es \cite{Basma2011} donde tambien se usan nodos de sensores magnéticos inalámbricos que se comunican con una estación base, la cual mediante algoritmos de predicción determina posibles colisiones y advierte a los conductores mediante un sistema de visualización en la infraestructura. Por otro lado, \cite{Dresner2008} y \cite{CondeBento2012} presentan aplicaciones en simulación para el control de las intersecciones usando agentes y aplicado en vehículos autónomos. En \cite{Quinlan2010} se hace una implementación en realidad aumentada del sistema propuesto por Dresner y Stone \cite{Dresner2008}. En esta implementación, los vehículos que se aproximan a la intersección solicitan autorización al controlador para ingresar y es este quien decide cuando pueden cruzar, dependiendo de las politicas de administración definidas.

%\subsection{Taxonomy}

\subsection{Developments in Intersection Management Systems}

In the table \ref{reviewtable} is presented a compilation of several developments within IMS field. For each work a brief comment is given, and a remark of how certain components (Data source, communication and application) are related to it. Target and implementation components could be infered from the comment.

\input{extra/ims_review_table}

\section{Sensor Fusion in Intersection Management} \label{s23}

Cameras are a wide used sensors for surveillance and monitoring applications because they provide a lot of information for scene understanding, but the transmitting and processing of such information requires high computational resources. For this reason low resolution cameras and low frame rates configurations are commonly deployed. On the other hand, range sensor, such as lasers and lidars, have been recently included in Intersection management systems, as unique sensors or with cameras too. One of the benefits of range sensors is that because of information volume is less than provided by cameras, its processing requirements are lower. Using both cameras and range sensors for intersection management systems makes it possible to take advantages of each type of sensor, and obtain a better representation of the environment.

Although many works based on cameras are found, there are two main developments that make use of range sensors along cameras to implement an IMS. The first one, developed by POSS research group at Peking University\footnote{http://www.poss.pku.edu.cn/index.html}, is the POSSi project, for which main objective is monitoring a dynamic environment through fusion of laser and video. In this project, a set of laser scanners are deployed on corners of an intersection and a camera is installed over a pedestrian bridge. They perform a fusion over laser data and after detection and recognition of vehicles and pedstrians, this info is backprojected on video stream.

The second relevant project, Ko-PER project\footnote{http://ko-fas.de/english/ko-per---cooperative-perception.html}, was developed by the Institute of Measurement, Control and Microtechnology at Ulm University under Ko-FAS research initiative. Ko-PER project aims to capture a complete picture of the local traffic environment. In order to do so, a full deployment of infrastructure was made, using 14 laserscanners, 10 cameras, a GPS and a V2I communication unit. Also, they perform a low-level fusion over same type sensor first, and then they merge outputs of each subsystem for improve detection and analysis results.

In the proposed work, there is not a deployment of sensors in an intersection scenario, but datasets will be used instead for validations purposes. Also, one of the objectives is to explore and propose a modular fusion architecture/framework for IMS based on cameras and laser sensors. In table \ref{summary_FusIMS} a summary of aforementioned developments is presented, including proposed work.

\begin{table}[!t]
\begin{tabular}{|p{0.13\linewidth}|c|p{0.125\linewidth}|p{0.15\linewidth}|p{0.16\linewidth}|c|}
\hline

\textbf{Reference} & \textbf{Project} & \textbf{Sensors} & \textbf{Target} & \textbf{Fusion} & \textbf{Framework} \\
\hline

\cite{Zhao2006} & POSSi & Laser & Vehicles, pedestrians & - & No \\
\hline

\cite{Song2008} & POSSi & Lasers & Pedestrians & Low-Level & No \\
\hline

\cite{Zhao2008} & POSSi & Lasers & Vehicles, pedestrians & Low-Level & No \\
\hline

\cite{Zhao2009} & POSSi & Lasers, Camera & Vehicles, pedestrians & Low-Level, High-Level & No \\
\hline

\cite{Zhao2012} & POSSi & Lasers & Vehicles, pedestrians & Low-Level & No \\
\hline

\cite{Goldhammer2012} & Ko-PER & Lasers, Cameras & Vehicles, pedestrians & Low-Level, Mid-Level & No \\
\hline

\cite{Meissner2012} & Ko-PER & Lasers & Pedestrians & Low-Level & No \\
\hline

\cite{Meissner2013} & Ko-PER & Lasers & Vehicles, pedestrians & Low-Level & No \\
\hline

\cite{Meissner2013a, Meissner2013b} & Ko-PER & Lasers, Cameras & Vehicles, pedestrians & Low-Level, Mid-Level & No \\
\hline

\cite{Strigel2013} & Ko-PER & Cameras & Vehicles & Low-Level & No \\
\hline

\cite{Meissner2014} & Ko-PER & Lasers, Cameras & Vehicles, pedestrians & Low-Level, Mid-Level & No \\
\hline

%\cite{} &  &  &  &  &  \\
%\hline

\textbf{Proposed work} & - & \textbf{Lasers, Cameras} & \textbf{Vehicles} & \textbf{Low-Level, Mid-Level, High-Level} & \textbf{Yes} \\
\hline

\end{tabular}
\caption{Summary of recent developments involving sensor fusion for Intersection Management Systems}
\label{summary_FusIMS}
\end{table}


\section{Conclusions}

Multisensor data fusion can be defined as the process of combine, merge or integrate data from homogeneous or heterogeneous set of sensors in order to get a better representation of a process, the environment or a situation, through the inference of underlying information and the improve of quality of data. Depending of the nature of the sensors and the source of information, fusion can be done in several manners, i.e., sensor-level fusion, feature-level fusion or decision-level fusion.

Several models, architectures and frameworks have been proposed in literature; some of them to show how a data fusion system should work and some other providing guidelines on how to implement it in a given application. Also, a wide range of algorithms have been used to perform fusion of data, some classified by the nature of the fusion or by the nature of the data, but is finally the environment and the application itself which determines the approach to use.

On the other hand, intersection management systems are highly required to improve safety and mobility in transportation, due to the high complexity present in these places for drivers and for authorities too. There is not an IMS application that address all the needs and problems in an intersection, for this reason many developments with narrow scopes and with many topics involved have been proposed through the years. To handle all these works and proposals, a new component-based classification scheme for IMS application has been proposed, and a review of developments in Intersection Management Systems is presented.


%\end{document}

% Architecture Description and system specifications
% 	* Multisensor IMS		
%		* Processing Stages
%		* Fusion approaches		
%	* Validation approaches
%		* Transportation and Traffic Simulations
%			* Simulators Classification
%		* Intersection monitoring Datasets
%			* POSSi Dataset
%			* KoPER Dataset
%	* Architecture proposal

\chapter [Architecture Description and System Specification]{Architecture Description and System Specification}
\chaptermark {System Description}

\epigraph{"Perfect is enemy of good"}{Voltaire}

Although several types of sensors are used for intersection monitoring and supervision, the use of cameras, lasers and lidars has increased due to advances in sensors manufacturing and computing capabilities. Such enhancements allows to deploy more of those types of sensors per scenario and it is required to define some processing stages from raw data capture through decision and control stages. It is also needed to test and validate the developments prior to a real and full functional implementation. The first part of this chapter describes the main stages in a intersection management system, from data preprocessing to situation assesment. Then, two validation tools for IMS applications are described, simulation models and datasets. And finally, the proposed architecture for the implementation of an IMS system is presented.

\section{Multisensor IMS}

Multicamera and multilasers monitoring systems offer more information about environment that can be merged to provide a better representation of the whole scene, detect with more accuracy the objects in the intersection, and prevent possible incidents. For designing a single-sensor or multi-sensor IMS, there are some basic processing stages to have into account. In the case of a multisensor system, it is also required to analyse and determine which is the better fusion approach to use and in which of the processing stages this fusion should be performed, in order to get better results than a single-sensor based system.

\subsection{Processing Stages}

\todo{Check stages description and figure}

In the designing of an IMS, there are four main stages that have to be performed from the data source to final output: preprocessing, feature analysis, pattern recognition and situation assessment. The aim of the first stage is to extract data of interest from the raw sensor information, using filtering to remove noise and irrelevant data, and background subtraction techniques to get the foreground of the scene. Spatial-temporal alignment of data is also performed in this stage. In the second stage, the objective is to identify elements within the foreground and extract relevant features of them. The third stage receives the set of features from the previous stage and performs recognition and classification tasks. Also, tracking and prediction of objects' state is performed based on historic information. In the fourth stage, object behaviour and inter-objects interaction are analysed to identify context and detect situations or events of interest. This output could be delivered to an automated or semi-automated stage of decision and control, to a human operator, or to a traffic agent or institution, to take immediate actions on traffic control, issue traffic tickets, warn drivers about possible incidents or improve transportation policies in a long-term basis. In figure \ref{proc_stages}, previously described stages are depicted, including a list of common tasks performed at each of these stages.

\begin{figure}[ht!]
\centering
\includegraphics[scale=0.55]{fig/3/processing_stages_and_tasks.png}
\caption{Processing stages in an IMS and commonly performed tasks within each of them}
\label{proc_stages}
\end{figure}


\subsection{Fusion approaches}

Depending on wheter the system has multiple sensors of the same type or different type, a data fusion approach should be chosen. When data came from same-type sensors, it is usually fused at low levels using techniques for temporal-spatial alignment, depending on sensors configuration. This is the case for a network of lasers or lidars or for multi-camera systems. If the system have different type of sensors, data from them may be fused at mid level, based on extracted features or classes on each subsystem; or may be fused at high level if each subsystem delivers control or decision outputs. The tasks shown in each of the processing stages (Figure \ref{proc_stages}), could be used as fusion blocks for homogeneous or heterogeneous data.

\section{Validation approaches}

Before deploying an IMS, it is needed to test some of its components and validate their results. Two commonly used tools for this purpose are traffic simulation models and datasets, depending on the component to be tested. Below, a description of each one is presented.

\subsection{Transportation and Traffic Simulation}

Transportation is a highly complex activity where different elements, like infrastructure, vehicles and pedestrians, affect efficiency, safety and quality of traffic. Intersections are special cases because there exists a high interaction between those mentioned elements making of these places critical points for mobility. For this reason, it is needed to have traffic models to allow the simulation of new policies or deployments intended to enhance transportation.

Three widely-known models for this kind of analysis are macroscopic model, microscopic model and mesoscopic model. The macroscopic model of traffic flow is based on a hydrodynamic analogy, modeling traffic as a fluid process characterized by three main variables, density, volume and speed, and the objective is to describe time-space evolution of those variables. The microscopic model aims to detail at a high detail inter-vehicles interactions and their individual state. For example, in a lane-change maneuver the state of the car doing the action and those affected by it, is individually tracked. The mesoscopic model tries to include features of both macroscopic and mesoscopic model. In this case, the same lane-change maneuver could be seen as an instant action triggered by lane density rather than on individual interactions.

\todo{Simulators description (?)}

A more detailed analysis of traffic simulation models and simulation platforms could be found in \cite{AdamsBoxill2000, Barcelo2000, Kitamura2005, Lieberman1992}




\subsection{Intersection monitoring Datasets} \label{possi_ds}

As mentioned in section \ref{s23}, POSS-i \footnote{Available at http://www.poss.pku.edu.cn/download.html} and Ko-PER \footnote{Available at http://www.uni-ulm.de/in/mrm/forschung/datensaetze.html} projects are leading the development of multisensor Intersection Management Systems. One of the contributions of these projects is the creation of datasets of such systems. They provide camera and laser information of a monitored intersection in Peking, China and Aschaffenburg, Germany, respectively.

%Next, a description of these two datasets will be given.
%
%\subsubsection{POSSi Dataset } \label{possi_ds}
%
%POSSi dataset\footnote{Available at http://www.poss.pku.edu.cn/download.html}
%
%\todo{Dataset description (?)}
%
%\subsubsection{KoPER Dataset }
%
%Ko-PER dataset\footnote{Available at http://www.uni-ulm.de/in/mrm/forschung/datensaetze.html}
%
%The full description of this dataset is presented in \cite{Strigel2014}.

\section{Architecture proposal}

The proposal presented in this work is based on MFI model processing entities in the sense that a processing block could take one or more inputs related between them and generate an output of the same type or a higher level output, that means that some blocks perform fusion while others just do some processing on incoming data. The communication or data exchange approach is based on JDL model, in which data is available over a "bus", where the processing blocks can write to or read from. And an information model is defined for abstract the elements of the scene, set a format for message exchange and define global and local configuration parameters. A general block diagram is shown in figure \ref{proposal_blocks}.

\begin{figure}[ht!]
\centering
\includegraphics[scale=0.7]{fig/3/proposal_blocks.pdf}
\caption{General overview of proposed architecture and its components. Red for the communication scheme, green for data model and blue for processing blocks}
\label{proposal_blocks}
\end{figure} 

\todo{Design Considerations}


\subsection{Communication scheme}

In order to isolate the processing tasks of the communication system, a Publisher-Subscriber approach is selected. The benefits of this option is that data can be shared between diferent blocks without generating a dependency from publishers to subscribers. Another benefit is that exposing a defined mechanism for publishing or subscribing, allows the system to be implementation agnostic.

Taking these reasons into account, Redis is chosen as communication platform. As stated in their website, "Redis is an open source (BSD licensed), in-memory data structure store, used as a database, cache and message broker"\cite{Redis} which support Publisher-Subscriber paradigm, allowing greater scalability and a more dynamic network topology.

For publishers and subscribers to exchange messages, channels are defined with labels using a namespace approach that describe the source of data and the kind of data. For example a channel containing raw data from a laser scanner could be labeled as \texttt{/sensors/range/laser\_1/data/raw}.

Redis is also used as in-memory storage for configuration data that is not part of the processing flow, like sensor parameters, intersection model and control options. This information is loaded into Redis from a configuration file formatted using JavaScript Object Notation (JSON), available for any client requiring it.
\todo{Define JSON HERE!!!!}

\subsection{Data Model}

The data model proposed for the system includes the abstraction of the sensing elements and the intersection model. For the sensing elements, a Sensor base class and two derived classes, RangeSensor class and ImageSensor class, are defined. 

The intersection has been modeled as a class with attributes like a label, a Map object, a set of RangeSensors and CameraSensors objects, a set of Leg objects, and a set of Area objects, for inner areas of the intersection. The Map class contains information about the geometry of the intersection and the coordinate system, including a image intended for visualization of configuration and processing.

The sensors sets are composed of zero or more sensor objects, as defined previously. An Area class is defined as having a bounding box and a label. From this class, a derived class, Leg, is defined for containing additional information about the legs of the intersection, such as heading of the leg, approaching or departure type. In figure \ref{data_model} is shown an UML diagram for the aforementioned classes and their relationships

\begin{figure}[ht!]
\centering
\includegraphics[scale=0.75]{fig/3/data_modelA.pdf}
\caption{UML diagram of defined entities and their relationships}
\label{data_model}
\end{figure}

The purpose of all the labels attributes within each class is to serve as an identifier of the channel in which the entity is publishing or subscribing to; in other words, to identify where is data generated from or where is data going to. For example, channel \texttt{/sensors/range/laser\_1/data/raw} contains raw data from a laser scanner labeled as \texttt{"laser\_1"} and \texttt{/legs/leg\_3/occupancy/} refers to the occupancy level of the leg labeled as \texttt{leg\_3}.

\subsection{Information Structure}

\todo{Move JSON definition from here!!!}
Data exchanged between processing blocks or any paramater stored in Redis, is formatted using JSON, allowing to use any JSON parser/formatter available in many languages, even it is possible to create simple scripts to interact with the system.

\subsubsection{.imscfg file}

For running, the systems requires first a .imscfg file which describes the scenario and the sensors configuration. This information is stored as a JSON object with the properties decribed in table \ref{imscfg_file}.

\begin{table}[ht!]
\footnotesize
\centering
\begin{tabular}{|c | c|}
\hline
\textbf{Property} & \textbf{Description} \\
\hline
name & Name of the configuration \\
\hline
map & Information of the scene, including map, coordinate system \\ 
 & and region of interest. \\
\hline
cameras & List of camera sensors information \\
\hline
range\_sensors & List of range sensors information \\
\hline
legs & List of legs information \\
\hline
intersection & Information of area of the intersection \\
\hline
\end{tabular}
\caption{Properties in .imscfg JSON file}
\label{imscfg_file}
\end{table}

\subsubsection{data messages and configuration parameters}

In addition to .imscfg file, there are different types of messages and parameters which have its own structure, for example used for sensors configuration or processed data. The table \ref{desc_map} gives a description of the types of information used.
% A more detailed documentation is given in appendix TODO.

\begin{table}[ht!]
\footnotesize
\centering
\begin{tabular}{|c | p{8cm}|}
\hline
\textbf{Name} & \textbf{Description} \\
\hline
laser\_pol\_msg & Contains a timestamp, an array of N angles and and array of N measurements \\
\hline
laser\_cart\_msg & Contains a timestamp, an array of N (x, y) coordinates \\
\hline
laser\_cfg & Contains position and orientation information, also has a background model for the laser sensor \\
\hline
occgrid\_msg & Contains a timestamp and an occupancy grid in the form of an MxN array \\
\hline
occgrid\_cfg & Contains occupancy grid configuration for the scene \\
\hline
clusters\_msg & Contains a set of clusters, each of them with an ID and a set of points \\
\hline

clusters\_cfg & Contains configuration about clustering algorithm and parameters \\
\hline

camera\_msg & Frames generated by camera \\
\hline

camera\_cfg & Information and parameters of the camera sensor \\
\hline

blobs\_msg & Bounding boxes detected as objects within a frame from the camera \\
\hline

blobs\_cfg & Parameters used by the image detection process\\
\hline

vs\_occ\_msg & Indicates the occupancy level of a virtual sensor, representing the entrace to or exit from the intersection\\
\hline

vs\_merge\_cfg & Parameters for configuring the merging process of different vs\_occ\_ms messages.\\
\hline

flow\_rate\_msg & Indicates the flow rate entering or exiting the intersection\\
\hline

flow\_cfg & Parameters for estimation of the flow rate, flow status and flow merging configuration\\
\hline

flow\_status\_msg & Indicates the flow status of an entrance or exit at intersection. This is a binary value, with value either "Traffic flowing" or "Traffic stopped".\\
\hline


\end{tabular}
\caption{Description of diferent types of messages and parameters used}
\label{desc_map}
\end{table}

%\todo{Finish table}


\subsection{Processing Blocks} \label{proc_blocks}

As stated in previous section, there are defined four stages in the processing flow, namely, preprocessing, feature analysis, pattern recognition and situation assessment. Within each of these stages there are different methods and techniques used to process the data and it is possible that those methods are suitable to perform fusion between homogeneous or heterogeneous data.

Below are detailed different processing blocks implemented as part of the whole architecture proposal, classified by the stage of processing in which are located in the process flow. Also, it is shown the format of data input, data output and parameters needed for configuration and execution.

Additionaly, there are included some tools and generic blocks used for dataset reading, data visualization and interactive control.

\subsubsection{Preprocessing}
\begin{description}
\item[laser\_background\_remove] \hfill

\begin{figure}[ht!]
\centering
\includegraphics[scale=1]{fig/3/laser_bg_remove.pdf}
\caption{laser\_bg\_remove}
\label{laser_bg_remove}
\end{figure}

This block takes as input a message of type laser\_pol\_msg, which contains raw readings from laser sensor. Also, it takes a laser\_cfg parameter which includes a background model for the laser sensor, generated from a set of scans and taking the maximum measurement in each angle of scanning, defined as follows:

Let suppose we have N scans from the laser and $d_{\theta i}$ be the distance measure at angle $\theta$ in scan $i$. Thus, the background value for that angle $\theta$ is $bg_\theta = max( \{d_{\theta i} | 1 < i < N\})$ and the background model is $bg = \{bg_\theta | \theta \in \Theta\} $ where $\Theta$ is the set of angles of scanning, in this case from -90\degree to 90\degree with 0.5\degree step.

\item[laser\_polar2cart] \hfill

\begin{figure}[ht!]
\centering
\includegraphics[scale=1]{fig/3/polar2cart.pdf}
\caption{polar\_to\_cartesian}
\label{polar_to_cartesian}
\end{figure}

This block takes as input a message of type laser\_pol\_msg for converting it into cartesian coordinates referenced to global system. For this reason, it also takes a laser\_cfg parameter to include laser position information in the conversion. Output message contains a set of x, y points obtained from the following equations:

\begin{eqnarray*}
O_x = I_\rho*\cos{\phi}+S_x \\
O_y = I_\rho*\sin{\phi} + S_y \\
\phi = I_\theta + S_\theta
\end{eqnarray*}

Where:

$(S_x, S_y, S_\theta)$: Position and orientation of the laser \\
$(I_\rho, I_\theta )$: Input message (distance and angle arrays) \\
$(O_x, O_y )$: Output message (x and y arrays) \\


\item[laser\_cart\_merge] \hfill

\begin{figure}[ht!]
\centering
\includegraphics[scale=1]{fig/3/laser_cart_merge.pdf}
\caption{laser\_cart\_merge}
\label{laser_cart_merge}
\end{figure}

This block takes as input messages of type laser\_cart\_msg coming from different sensors and merges them into a single message, based on a sampling period $T$.

%\item[camera\_background\_remove] \hfill
%
%\begin{figure}[ht!]
%\centering
%\includegraphics[scale=1]{fig/3/camera_bg_remove.pdf}
%\caption{camera\_background\_remove block}
%\label{cam_bg_remove}
%\end{figure}

\end{description}


\subsubsection{Feature Analysis}
\begin{description}
\item[points\_to\_cluster] \hfill

\begin{figure}[ht!]
\centering
\includegraphics[scale=1]{fig/3/points_to_clusters.pdf}
\caption{points\_to\_clusters}
\label{points_to_clusters}
\end{figure}

The input message for this block is a set of points corresponding to objects scanned by lasers. Clustering is performed over this input to identify the points belonging to the same object. The algorithm used in this implementation is DBSCAN, which stands for Density-Based Spatial Clustering of Applications with Noise. This algorithm does not need an estimated number of clusters as input, instead of this, it requires only two parameters: a minimum number of points per cluster, m, and a neighbourhood measure, $\epsilon$. A detailed description of the algorithm, can be found in \cite{Ester96}.

As output, a message of type clustering\_msg, containing information like clusters ID and set of points belonging to each cluster, is delivered.

\item[points\_to\_occgrid] \hfill

\begin{figure}[ht!]
\centering
\includegraphics[scale=1]{fig/3/points_to_grid.pdf}
\caption{points\_to\_grid}
\label{points_to_grid}
\end{figure}

This block generates a occupancy grid based on cartesian data from laser sensors. The cartesian data received from sensors is merged in a timeslot basis, filling a buffer for each sensor and processing available data using a defined period $T_{om}$. As parameters, this block receives a cell size for the grid, the laser sensors configuration and the scene configuration. The grid size is of $N_C$ Columns and $N_R$ rows, obtained as follows:

\begin{eqnarray*}
X = x_{max}-x_{min} \\
Y = y_{max}-y_{min} \\
N_C = ceil(X/C)+1 \\
N_R = ceil(Y/C)+1 \\
\end{eqnarray*}

Where $(x_{min}, y_{min})$ and $(x_{max}, y_{max})$ are the bottom-left and top-right coordinates of the region of interest in the scene, respectively.

Each cell of the grid will store a value indicating a probability of that cell being occupied. Initially, all cells have a value of $0.5$. In order, to update the grid using data from laser, the points corresponding to laser position and measures, should be located in a cell of the grid, and then mark it as occupied. The following equations show how a point $P = (P_x, P_y)$ is tranformed to a cell $C_{ij}$, where $i$ is the column and $j$ is the row in the grid:

\begin{eqnarray*}
i = ceil((P_x - x_{min})/C - 0.5) \\
j = N_R - ceil((P_y - y_{min})/C - 0.5) \\
\end{eqnarray*}

Then, the cells that form a straigh line between the cell of laser position and the cells of each laser measure, should be marked as empty. To get the list of these cells Bresenham's algorithm is used. This algorithm is widely used in computer graphics due to it simplicity and because it uses integer arithmetics, making it computationally cheap. The input of this algorithm is a pair of coordinates, the endpoints of the line, and as result, the set of points that completes the line between, is returned.
%\todo{Check bresenham algorithm reference}

As the output, this block generates an occupancy grid of the scene, which is published at a rate defined by the previous period $T_{om}$.
%\todo{update policy}


\item[camera\_blobs] \hfill

\begin{figure}[ht!]
\centering
\includegraphics[scale=1]{fig/3/camera_blobs.pdf}
\caption{camera\_blobs}
\label{camera_blobs}
\end{figure}

The purpose of this block is to detect the vehicles from the streaming of video from a camera. To accomplish this task, the YOLOv3 detection system \cite{yolov3} was wraped into a block. The algorithm used by this detector, applies a single neural network to the full image. This network divides the image into regions and predicts bounding boxes and probabilities for each region. These bounding boxes are weighted by the predicted probabilities. The output of this block, are the bounding boxes with high probabilty of being a vehicle.


\item[camera\_blobs2occgrid] \hfill

\begin{figure}[ht!]
\centering
\includegraphics[scale=1]{fig/3/camera_blobs2occgrid.pdf}
\caption{camera\_blobs2occgrid}
\label{camera_blobs2occgrid}
\end{figure}

This block receives bounding boxes corresponding to vehicles and transform them from image coordinate system to the global reference system, Then, it maps the area covered by the boxes into an occupancuy grid in order to mark those cells as occupied in the same fashion as described for the block "points\_to\_occgrid" \ref{points_to_grid}.

\end{description}

\subsubsection{Pattern Recognition}
\begin{description}

\item[virtual\_sensor\_occupancy] \hfill

\begin{figure}[ht!]
\centering
\includegraphics[scale=1]{fig/3/vs_occ.pdf}
\caption{Virtual\_sensor\_occupancy}
\label{vehicle_counter}
\end{figure}

This block accepts different types of inputs, cluster\_msg and occgrid\_msg. It process the incoming data, and generate bounding boxes based on the clusters or based on the occupancy grid, according to the input. Having these bounding boxes, which represent detected vehicles, scene configuration data is loaded and the geometric parameters of the legs and lanes are used to generate virtual sensors at the entry and exit points of the intersection and check the overlaping of detected vehicles with aforemetioned virtual sensors. The output generated by this block is the occupancy percentage of the virtual sensor, represented as a message of type vs\_occ\_msg.
%vehicle counter $v_{cnt}$ is incremented by one when a transition from occupied state to empty state is detected in a new frame.


\begin{description}

\item[virtual\_sensor\_merge] \hfill
\begin{figure}[ht!]
\centering
\includegraphics[scale=1]{fig/3/vs_merge.pdf}
\caption{virtual\_sensor\_merge}
\label{virtual_sensor_merge}
\end{figure}

This block is intended to take as inputs different messages of type vs\_occ\_msg and merging them to increase the reliabilty of detection. First, a moving average is calculated over each input to reduce the effect of false positives or false negatives in a single frame. After this, the inputs are merged by a defined method like average, weighted average, maximum or minimum. Then, a new vs\_occ\_msg message is dispatched as output.

\end{description}


\subsubsection{Situation Assesment}

\item[virtual\_sensor\_to\_flow\_rate] \hfill
\begin{figure}[ht!]
\centering
\includegraphics[scale=1]{fig/3/vs2flow.pdf}
\caption{virtual\_sensor\_to\_flow\_rate}
\label{virtual_sensor_to_flow_status}
\end{figure}

When a message of type vs\_occ\_msg arrives, it is evaluated using a threshold $vo_{th}$, if the occupancy percentage of a virtual sensor is greater than $vo_{th}$, it is considered that there is a vehicle over the sensor in that frame, this is defined as occupied state. This detection is made in a frame basis, thus, a vehicle counter $v_{cnt}$ is incremented by one when a transition from occupied state to empty state is detected in a new frame.

Then, a time interval $t_{c}$ is defined for calculating the flow rate $v_{r}$. Defined as the number of transitions during that interval divided by $t_{c}$. The output message type is flow\_rate\_msg.

\item[flow\_rate\_to\_flow\_status] \hfill
\begin{figure}[ht!]
\centering
\includegraphics[scale=1]{fig/3/flow2status.pdf}
\caption{flow\_rate\_to\_flow\_status}
\label{flow_rate_to_flow_status}
\end{figure}


This block process one or more inputs of type flow\_rate\_msg. If there are more than one input, it merges them using defined policy, like average or maximum. Then, if the flow rate $v_{r}$ is above a flow threshold $f_{th}$, it is considered that the traffic is flowing. Otherwise, traffic is considered stopped.

\end{description}
\todo{Check formatting here}
\subsubsection{Tools and Utilities}
\begin{description}

\item[laser\_publisher] \hfill
\begin{figure}[ht!]
\centering
\includegraphics[scale=1]{fig/3/laser_publisher.pdf}
\includegraphics[scale=1]{fig/3/camera_publisher.pdf}
\caption{laser\_publisher and camera\_publisher}
\label{sensors_publishers}
\end{figure}

These blocks are intended to read dataset files containing the information from the sensors and publish this data with the appropriate format and preserving the original rate of the sensors, if needed.

%\item[camera\_publisher] \hfill
%\begin{figure}[ht!]
%\centering
%\includegraphics[scale=1]{fig/3/camera_publisher.pdf}
%\caption{camera\_publisher}
%\label{camera_publisher}
%\end{figure}

%\item[dataset\_control] \hfill
%\begin{figure}[ht!]
%\centering
%\includegraphics[scale=1]{fig/3/dataset_ctrl.pdf}
%\caption{dataset\_control}
%\label{dataset_control}
%\end{figure}

\item[data\_viewer] \hfill
\begin{figure}[ht!]
\centering
\includegraphics[scale=1]{fig/3/data_viewer.pdf}
\caption{data\_viewer}
\label{data_viewer}
\end{figure}

The objective of this block is to allow the visualisation of the data in any of the processing stages. It has the option to display and XY coordinate system, an occupancy grid or a map of the intersection overlaped with detections or metrics.

\end{description}

\section{Conclusions}

A whole intersection management system contains many different components, for example the instrumentation of the junction, the handling of data and information, the control policies and requirements, and the interaction between those elements. None of these items is more relevant than the others and it is for this reason that each one of them should be well conceived having in mind a common framework that allows their scalability, modularity and reliability. When these three features are accomplished, a change or an upgrade of any part of the system will not affect the remaining ones.

It is also important that an Intersection Management System should be flexible, not tied to one deployment or scenario, but also that allows to perform tests over new datasets or junctions, at no extra cost or just with some configuration adjustments. Although it is hard to have a one-rules-them-all system due to the variety of intersections (crossroads with different numbers of legs, roundabouts, T-shaped, Y-shaped, etc.), there are common needs, features and requirements that makes relevant to use an interesection management system looking forward to monitor, control and improve safety and efficiency in transportation.

\chapter [Deployment and Results]{Deployment and Results}

\epigraph{"Testing leads to failure, and failure leads to understanding"}{Burt Rutan}

In this chapter is presented the experimental setup used to deploy and test the proposed architecture. First is described the data and hardware and software specifications. Then, two implemented configurations of the system are presented. Finally, results are summarized and analysed.

\section{Experimental Setup}

\begin{figure}[ht!]
\centering
\includegraphics[scale=0.8]{fig/4/intersection-config2.pdf}
\caption{Possi dataset configuration}
\label{possi_img}
\end{figure}

\begin{figure}[ht!]
\centering
\includegraphics[scale=0.2]{fig/4/sensors_overview.jpeg}
\caption{Overview of sensors data. Left: Laser scanners, Right: Camera.}
\label{possi_sensors}
\end{figure}

%\subsection{Data}
The selected dataset for testing the system is the one from POSSi project (section \ref{possi_ds}). It contains 6 laser-scanners raw readings and a video from a camera located over the intersection. Figure \ref{possi_img} depicts the configuration used by this dataset. In figure \ref{possi_sensors} is shown the data from each type of sensors, laser scanners and camera. The ground truth data used for validation is the vehicle count over 3 of the 5 legs of the intersection, taking into account the time at which a vehicle appears. This groundtruth data was generated manually.

 
%\subsection{Hardware and software specifications}

The platform used for the implementation is a laptop ASUS GL552VW, with 8GB RAM DDR4, a processor Intel Core i7 6700HQ @ 2.60GHz x 8 cores and graphic card Nvidia GeForce GTX 960M. The operating systems running on it is Ubuntu 16.04 LTS. The software was developed using primarily Python3 and C++ as programming languages. Used libraries are MatplotLib, Scikit, Numpy, OpenCV, and darknet for data processing and visualisation. For communication scheme redis was installed along some JSON parsers for formatting.

\section{Test Configurations}

\begin{table}[ht!]
\footnotesize
\centering
\begin{tabular}{|c | c| c|}
\hline
\textbf{Proc. Stage} & \textbf{Block ID} & \textbf{Name} \\
\hline

\multirow{2}{*}{Data feeding} &
A & camera\_publisher \\
\cline{2-3} 
& B & laser\_publisher \\
\hline

\multirow{3}{*}{Preprocessing} &
C & laser\_bg\_remover \\
\cline{2-3}
& D & laser\_pol2cart \\
\cline{2-3}
& E & laser\_cart\_merge \\
\hline

\multirow{4}{*}{Feature analysis} &
F & points2clusters \\
\cline{2-3}
& G & points2occgrid \\
\cline{2-3}
& H & camera\_blobs \\
\cline{2-3}
& I & camera\_blobs2occgrid \\
\hline

\multirow{2}{*}{Pattern recognition} &
J & virtual\_sensor\_occupancy \\
\cline{2-3}
& K & virtual\_sensor\_merge \\
\hline

\multirow{2}{*}{Situation Assesment} &
L & virtual\_sensor\_to\_flow\_rate \\
\cline{2-3}
& M & flow\_rate\_to\_flow\_status \\
\hline

\end{tabular}
\caption{Description of processing blocks used in test configurations}
\label{desc_test_config}
\end{table}

In order to test the proposed architecture, two different configurations have been selected. The first configuration is using just the camera information. The second configuration is based on multiple laser sensors along with the camera.

Each configuration consists of a set of processing blocks (as described in section \ref{proc_blocks}) connected, aiming to take data from  lasers and cameras, to produce an output of higher level. In this case, the binary status of a leg (traffic flowing, traffic stopped) is the metric used for evaluation, derived from the vehicle count.

The following table lists all used blocks and assigns an ID for each one. Those IDs are used in the graph description in its own section. Also, bold nodes and connections indicate multiple instance of the same element. The figure \ref{tconf1} shows the  configuration used for processing just video data. Figure \ref{tconf2} shows the configuration when a set of lasers is added to the single-camera configuration. The blue line represents the fusion at node K, which is fusion on virtual sensors level. The purple line represents the fusion at flow rate level.

\begin{figure}[ht!]
\centering
\includegraphics[scale=0.6]{fig/4/test_configuration1.pdf}
\caption{Single camera configuration}
\label{tconf1}
\end{figure}

\begin{figure}[ht!]
\centering
\includegraphics[scale=0.6]{fig/4/test_configuration2.pdf}
\caption{Multiple lasers and camera configuration}
\label{tconf2}
\end{figure}

%TODO

%\subsection{Case 3: Multiple Lasers and camera}
%
%\begin{figure}[ht!]
%\centering
%\includegraphics[scale=0.7]{fig/4/test_configuration3.pdf}
%\caption{Multiple lasers and camera configuration}
%\label{tconf3}
%\end{figure}

%TODO

\section{Results}

\subsection{Case 1: Single camera}

As stated before, the flow status is used for comparing the system with the ground truth. This status is derived from the flow rate in vehicles per second. If this value is greater than a defined threshold, it is considerded to be in "traffic flowing" status.

As result of the first stages of processing, vehicle detection in three diferent frames are shown in figure \ref{camera_detection}. This is a representation of the output from node H, camera\_blobs.

\begin{figure}[ht!]
\centering
\includegraphics[scale=0.11]{fig/4/camera.jpeg}
\caption{Vehicle detection using camera}
\label{camera_detection}
\end{figure}

In the figure \ref{video_res} is shown the comparison between the flow profile obtained from the system and the generated by the ground truth. There is also remarked one of the intervals where the output is "traffic flowing". After evaluating the performance of the system in a frame basis, a confusion matrix is obtained (table \ref{video_cf}), here $F$ denotes the state "traffic Flowing" and $S$ denotes "traffic Stopped" and we get an accuracy of 91.82\% and a TPR (True Positive Rate) of 55.70\%.

\begin{table}[ht!]
\footnotesize
\centering
\noindent
\renewcommand\arraystretch{1.5}
\setlength\tabcolsep{0pt}
\begin{tabular}{c c c c c}
  \multirow{10}{*}{\rotatebox{90}{\parbox{1.1cm}{\bfseries\centering Actual\\ value}}} & 
    & \multicolumn{2}{c}{\bfseries Outcome} & \\
  & & \bfseries F & \bfseries S & \bfseries total \\
  & F$'$ & \MyBox{694}{} & \MyBox{552}{} & 1246 \\[2.4em]
  & S$'$ & \MyBox{26}{} & \MyBox{5791}{} & 5817 \\
  & total & 720 & 6343 &
\end{tabular}
\caption{Confusion matrix of the system based on video}
\label{video_cf}
\end{table}


\begin{figure}[ht!]
\centering
\includegraphics[scale=0.45]{fig/4/video_res.pdf}
\caption{Results for the single camera implementation in one of the legs of the intersection}
\label{video_res}
\end{figure}


\subsection{Case 2: Camera and multiple lasers}
%{'TN': 5791, 'TP': 694, 'FP': 26, 'FN': 552}
%v30 FPR: 0.0045, F1: 0.7060, NPV: 0.5108, reca: 0.5570, spec: 0.9955, PPV: 0.9639, accu: 0.9182, prec: 0.9639, TPR: 0.5570, prev: 0.1764, 
%{'TN': 5650, 'TP': 23, 'FP': 167, 'FN': 1223}
%l30 FPR: 0.0287, F1: 0.0320, NPV: 0.0156, reca: 0.0185, spec: 0.9713, PPV: 0.1211, accu: 0.8032, prec: 0.1211, TPR: 0.0185, prev: 0.1764, 
%{'TN': 5414, 'TP': 1098, 'FP': 403, 'FN': 148}
%vl_avg_1_30 FPR: 0.0693, F1: 0.7994, NPV: 0.9216, reca: 0.8812, spec: 0.9307, PPV: 0.7315, accu: 0.9220, prec: 0.7315, TPR: 0.8812, prev: 0.1764, 
%{'TN': 5669, 'TP': 792, 'FP': 148, 'FN': 454}
%vl_max_1_30 FPR: 0.0254, F1: 0.7246, NPV: 0.6039, reca: 0.6356, spec: 0.9746, PPV: 0.8426, accu: 0.9148, prec: 0.8426, TPR: 0.6356, prev: 0.1764, 
%{'TN': 5804, 'TP': 161, 'FP': 13, 'FN': 1085}
%vl_f_avg_30 FPR: 0.0022, F1: 0.2268, NPV: 0.1091, reca: 0.1292, spec: 0.9978, PPV: 0.9253, accu: 0.8445, prec: 0.9253, TPR: 0.1292, prev: 0.1764, 
%{'TN': 5624, 'TP': 716, 'FP': 193, 'FN': 530}
%vl_f_max_30 FPR: 0.0332, F1: 0.6645, NPV: 0.5432, reca: 0.5746, spec: 0.9668, PPV: 0.7877, accu: 0.8976, prec: 0.7877, TPR: 0.5746, prev: 0.1764,

%v30         reca: 0.5570, spec: 0.9955, accu: 0.9182, prec: 0.9639, TPR: 0.5570,
%vl_avg_1_30 reca: 0.8812, spec: 0.9307, accu: 0.9220, prec: 0.7315, TPR: 0.8812,
%vl_max_1_30 reca: 0.6356, spec: 0.9746, accu: 0.9148, prec: 0.8426, TPR: 0.6356,
%vl_f_avg_30 reca: 0.1292, spec: 0.9978, accu: 0.8445, prec: 0.9253, TPR: 0.1292,
%vl_f_max_30 reca: 0.5746, spec: 0.9668, accu: 0.8976, prec: 0.7877, TPR: 0.5746

In this test case, the information from the laser scanners is used with the purpose of increase the performance of the system. This required just to include the appropiate nodes for laser data processing, and replace some nodes from the video processing graph to allow data fusion from laser and cameras. An example of vehicle detection based on laser scanners is shown in figure \ref{laser_detection}. This is the output of the node F, points2clusters.

\begin{figure}[htb!]
\centering
\includegraphics[scale=0.4]{fig/4/laser1a.png}
\caption{Vehicle detection using laser scanners}
\label{laser_detection}
\end{figure}

For merging information from laser and cameras, it was decided to perform the fusion at two levels, at the virtual sensor occupancy level and at the flow rate level. Both of these approaches were implemented using average and maximum policies. In the figure \ref{vl_res} is depicted the flow rate output for all of four configurations.

Performing the same frame-based analysis as for the video system, we obtain the confusion matrices presented in table \ref{vl_cf}. It is noted that performing the fusion at virtual sensor occupancy level, the accuracy of the systems has almost no change, 92.2\% and 91.48\% for average and maximum policies respectively. For the TPR, there is a significative increase, having 88.12\% and 63.56\%, also for average and maximum policies.

When the fusion is performed at flow rate level using average policy, the accuracy and TPR values are reduced to 84.45\% and 12.92\%. Using maximum policy, there is a small decrement in the values, with an accuracy of 89.76\% and a TPR of 57.46\%.

\begin{figure}[htb!]
\centering
\includegraphics[scale=0.5]{fig/4/vl_res.pdf}
\caption{Results for the multisensor implementation using four different fusion approaches}
\label{vl_res}
\end{figure}

\begin{table}[htb!]
\footnotesize
\centering
\noindent
\renewcommand\arraystretch{1.5}
\setlength\tabcolsep{0pt}
\begin{tabular}{c c c c c}
  \multirow{10}{*}{\rotatebox{90}{\parbox{1.1cm}{\bfseries\centering Actual\\ value}}} & 
     \multicolumn{4}{c}{\bfseries Outcome (avg vs\_occ)} \\
  & & \bfseries F & \bfseries S & \bfseries total \\
  & F$'$ & \MyBox{1098}{} & \MyBox{148}{} & 1246 \\[2.4em]
  & S$'$ & \MyBox{403}{} & \MyBox{5414}{} & 5817 \\
  & total & 1501 & 5562 &
\end{tabular}

\begin{tabular}{c c c c c}
  \multirow{10}{*}{\rotatebox{90}{\parbox{1.1cm}{\bfseries\centering Actual\\ value}}} & 
     \multicolumn{4}{c}{\bfseries Outcome (max vs\_occ)}  \\
  & & \bfseries F & \bfseries S & \bfseries total \\
  & F$'$ & \MyBox{792}{} & \MyBox{454}{} & 1246 \\[2.4em]
  & S$'$ & \MyBox{148}{} & \MyBox{5669}{} & 5817 \\
  & total & 940 & 6123 &
\end{tabular}

\begin{tabular}{c c c c c}
  \multirow{10}{*}{\rotatebox{90}{\parbox{1.1cm}{\bfseries\centering Actual\\ value}}} & 
     \multicolumn{4}{c}{\bfseries Outcome (avg flow)} \\
  & & \bfseries F & \bfseries S & \bfseries total \\
  & F$'$ & \MyBox{161}{} & \MyBox{1085}{} & 1246 \\[2.4em]
  & S$'$ & \MyBox{13}{} & \MyBox{5804}{} & 5817 \\
  & total & 174 & 6889 &
\end{tabular}

\begin{tabular}{c c c c c}
  \multirow{10}{*}{\rotatebox{90}{\parbox{1.1cm}{\bfseries\centering Actual\\ value}}} & 
     \multicolumn{4}{c}{\bfseries Outcome (max flow)}  \\
  & & \bfseries F & \bfseries S & \bfseries total \\
  & F$'$ & \MyBox{716}{} & \MyBox{530}{} & 1246 \\[2.4em]
  & S$'$ & \MyBox{193}{} & \MyBox{5624}{} & 5817 \\
  & total & 909 & 6154 &
\end{tabular}

\caption{Fusion configurations from top to bottom: Average on virtual sensor occupancy, Maximum on virtual sensor occupancy, Average on flow rate, Maximum on flow rate}
\label{vl_cf}
\end{table}

\section{Conclusions}

An intersection management system is a complex set of subsystems, and for a feasible implementation and deployment as a real solution it is needed for it to be easily tested, adapted and enhanced. In this case, after having a well-concieved block design, it was relatively straight forward to start building a video based system for getting an indicator about traffic flow in an intersection using a dataset.

For stakeholders of the system outcome, high level indicators are desired, like traffic status, no matter which low-level hardware or software are used. For this reason, modularity and definitions on data and messages types, allow development of new features without compromising the integrity of the system. Also, new sources of information and data could be easily integrated if the processing of them is made according to the definitions. In that way, it was possible to integrate laser scanner data provided by the dataset, taking into account to generate outputs already defined by the messages specification.

For different sources of information it is not enough to perform a fusion strategy to obtain better results. It is needed to analyse the type of data and the level at which fusion should be performed. Also, it is important to understand if the fusion is competitive, complementary or cooperative, in order to select the appropiate strategy and increase the performance of the system. An example of this, was found when fusing data from camera and lasers at two different levels and using different strategies. It was clear from the results that altough the data was the same, one configuration increased the performace while another one decrease it.







%ch5-conclusions
\chapter {Conclusions}

\section{General Conclusions}
\section{Contributions}

A scalable, modular system for traffic control monitoring at intersection. Both for experimentation and real deployment.

Set of software repositories for application and for available datasets containing intersection data.

Pair of articles submmited to events, one for architecture description and example implementation and another for application comparison. 

\section{Future Work}

Generation of our own dataset.

Optimization of communication scheme.

Integration of new types of sensors

Creation of a more friendly user interface.


%\begin{appendices}
%	\chapter {Vehicular Environment Simulators}
%\end{appendices}


\bibliography{bibliography}
%\bibliography{ref}



\end{document}
