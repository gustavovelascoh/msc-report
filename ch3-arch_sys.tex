% Architecture Description and system specifications
% 	* Multisensor IMS		
%		* Processing Stages
%		* Fusion approaches		
%	* Validation approaches
%		* Transportation and Traffic Simulations
%			* Simulators Classification
%		* Intersection monitoring Datasets
%			* POSSi Dataset
%			* KoPER Dataset
%	* Architecture proposal

\chapter [Architecture Description and System Specification]{Architecture Description and System Specification}
\chaptermark {System Description}

Although several types of sensors are used for intersection monitoring and supervision, the use of cameras, lasers and lidars has increased due to advances in sensors manufacturing and computing capabilities. Such enhancements allows to deploy more of those types of sensors per scenario and it is required to define some processing stages from raw data capture through decision and control stages. It is also needed to test and validate the developments prior to a real and full functional implementation. The first part of this chapter describes the main stages in a intersection management system, from data preprocessing to situation assesment. Then, two validation tools for IMS applications are described, simulation models and datasets. And finally, the proposed architecture for the implementation of an IMS system is presented.

\section{Multisensor IMS}

Multicamera and multilasers monitoring systems offer more information about environment that can be merged to provide a better representation of the whole scene, detect with more accuracy the objects in the intersection, and prevent possible incidents. For designing a single-sensor or multi-sensor IMS, there are some basic processing stages to have into account. In the case of a multisensor system, it is also required to analyse and determine which is the better fusion approach to use and in which of the processing stages this fusion should be performed, in order to get better results than a single-sensor based system.

\subsection{Processing Stages}

In the designing of an IMS, there are four main stages that have to be performed from the data source to final output: preprocessing, feature analysis, pattern recognition and situation assessment. The aim of the first stage is to extract data of interest from the raw sensor information, using filtering to remove noise and irrelevant data, and background subtraction techniques to get the foreground of the scene. Spatial-temporal alignment of data is also performed in this stage. In the second stage, the objective is to identify elements within the foreground and extract relevant features of them. The third stage receives the set of features from the previous stage and performs recognition and classification tasks. Also, tracking and prediction of objects' state is performed based on historic information. In the fourth stage, object behaviour and inter-objects interaction are analysed to identify context and detect situations or events of interest. This output could be delivered to an optional fifth stage of decision and control, to a human operator, or to a traffic agent or institution, to take immediate actions on traffic control, issue traffic tickets, warn drivers about possible incidents or improve transportation policies in a long-term basis. In figure \ref{proc_stages}, previously described stages are depicted, including common task performed at each of these stage and also is shown how the data volume is reduced while data meaning increases in the last stages. 

\begin{figure}[ht!]
\centering
\includegraphics[scale=0.55]{fig/3/processing_stages_and_tasks.png}
\caption{Dataflow through processing stages in an IMS an commonly performed tasks}
\label{proc_stages}
\end{figure}


\subsection{Fusion approaches}

Depending on wheter the system has multiple sensors of the same type or different type, a data fusion approach should be chosen. When data came from same-type sensors, it is usually fused at low levels using techniques for temporal-spatial alignment, depending on sensors configuration. This is the case for a network of lasers or lidars or for multi-camera systems. If the system have different type of sensors, data from them may be fused at mid level, based on extracted features or classes on each subsystem; or may be fused at high level if each subsystem delivers control or decision outputs. The tasks shown in each of the processing stages (Figure \ref{proc_stages}), could be used as fusion blocks for homogeneous or heterogeneous data.

\section{Validation approaches}

Before deploying an IMS, it is needed to test some of its components and validate their results. Two commonly used tools for this purpose are traffic simulation models and datasets, depending on the component to be tested. Below, a description of each one is presented.

\subsection{Transportation and Traffic Simulation}

Transportation is a highly complex activity where different elements, like infrastructure, vehicles and pedestrians, affect efficiency, safety and quality of traffic. Intersections are special cases because there exists a high interaction between those mentioned elements making of these places critical points for mobility. For this reason, it is needed to have traffic models to allow the simulation of new policies or deployments intended to enhance transportation.

Three widely-known models for this kind of analysis are macroscopic model, microscopic model and mesoscopic model. The macroscopic model of traffic flow is based on a hydrodynamic analogy, modeling traffic as a fluid process characterized by three main variables, density, volume and speed, and the objective is to describe time-space evolution of those variables.


A more detailed analysis of traffic simulation models and simulation platforms could be found in \cite{AdamsBoxill2000, Barcelo2000, Kitamura2005, Lieberman1992}
\subsubsection{Simulators Classification}
\subsection{Intersection monitoring Datasets}

As mentioned in section \ref{s23}, POSS-i and Ko-PER projects are leading the development of multisensor Intersection Management Systems. One of the contributions of these projects is the creation of datasets of such systems. They provide camera and laser information of a monitored intersection in Peking, China and Aschaffenburg, Germany, respectively. Next, a description of these two datasets will be given.

\subsubsection{POSSi Dataset }

POSSi dataset\footnote{Available at http://www.poss.pku.edu.cn/download.html}
\subsubsection{KoPER Dataset }

Ko-PER dataset\footnote{Available at http://www.uni-ulm.de/in/mrm/forschung/datensaetze.html}

The full description of this dataset is presented in \cite{Strigel2014}.

\section{Architecture proposal}

The proposal presented in this work is based on MFI model processing entities in the sense that a processing block could take one or more inputs related between them and generate an output of the same type or a higher level output. The communication or data exchange approach is based on JDL model, in which data is available over a "bus", where the processing blocks can write to or read from. Three main components of the architecture are defined: communication scheme, data model and  processing blocks. 



\subsection{Communication scheme}

In order to isolate the processing tasks of the communication system, a Publisher-Subscriber approach is selected. The benefits of this option is that data can be shared between diferent blocks without generating a dependency from publishers to subscribers. Another benefit is that exposing a defined mechanism for publishing or subscribing, allows the system to be implementation agnostic.

Taking these reasons into account, Redis is chosen as communication platform. As stated in their website, "Redis is an open source (BSD licensed), in-memory data structure store, used as a database, cache and message broker"\cite{Redis} which support Publisher-Subscriber paradigm, allowing greater scalability and a more dynamic network topology.

For publishers and subscribers to exchange messages, channels are defined with labels using a namespace approach that describe the source of data and the kind of data. For example a channel containing raw data from a laser scanner could be labeled as "/sensors/range/laser\_N/data/raw".

Redis is also used as in-memory storage for configuration data that is not part of the processing flow, like sensor parameters, intersection model and control options. This information is loaded into Redis from a JSON file and it is available for any client requiring it.


\subsection{Data Model}

The data model proposed for the system includes the abstraction of the sensing elements, the intersection model and the data used and generated by the processing blocks. For the sensing elements, a Sensor base class and two derived classes, Laser class and Camera class, are defined.

---TODO: Describe and add UML of classes --
-- describe intersection model
--Describe data from flow



\subsection{Processing Blocks}