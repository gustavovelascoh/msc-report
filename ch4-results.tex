\chapter [Deployment and Results]{Deployment and Results}

\section{Experimental Setup}

The selected dataset testing the system is POSSi's \ref{possi_ds}. It contains 6 laser-scanners raw readings and a video from a camera located over the intersection. The ground truth data used for validation is the vehicle count over 3 of the 5 legs of the intersection, taking into account the time at which a vehicle appears. This groundtruth data was generated by human inspection.
 

\section{Test Configurations}
In order to test the proposed architecture, three different configurations have been selected. The first configuration is using just one laser sensor. The second configuration is based on multiple laser sensors, Finally, the third configuration uses multiple lasers along with a camera.

Each configuration consists of a set of processing blocks (as described in section \ref{proc_blocks}) connected, aiming to take data from sensors, lasers and cameras, to produce an output of higher level. In this case, vehicle counting is the metric used for evaluation.

The following table lists all used blocks and assigns an ID for each one. Those IDs are used in the graph description in its own section. Also, bold nodes and connections indicate multiple instance of the same element.

\begin{table}[ht!]
\footnotesize
\centering
\begin{tabular}{|c | c| c|}
\hline
\textbf{Block ID} & \textbf{Name} \\
\hline
A & laser\_publisher \\
\hline
B & laser\_bg\_remover \\
\hline
C & laser\_pol2cart \\
\hline
D & laser\_cart\_merge \\
\hline
E & points2clusters \\
\hline
F & clusters\_veh\_counter \\
\hline
G & veh\_counter\_merge \\
\hline
H & clusters2occ\_level \\
\hline
I & occ\_level\_merge \\
\hline
J & laser\_pol2cart \\
\hline
K & points2occgrid \\
\hline
L & occgrid\_merge \\
\hline
M & occgrid\_leg\_counter \\
\hline
N & occgrid2occ\_level \\
\hline
O & camera\_publisher \\
\hline
P & camera\_bg\_remover \\
\hline
Q & camera\_blobs \\
\hline
R & camera\_blobs2occgrid \\
\hline
\end{tabular}
\caption{Description of processing blocks used in test configurations}
\label{desc_test_config}
\end{table}


\subsection{Case 1: Single Laser}

\begin{figure}[ht!]
\centering
\includegraphics[scale=0.7]{fig/4/test_configuration1.pdf}
\caption{Single laser configuration}
\label{tconf1}
\end{figure}

TODO


\subsection{Case 2: Multiple Lasers}

\begin{figure}[ht!]
\centering
\includegraphics[scale=0.7]{fig/4/test_configuration2.pdf}
\caption{Multiple lasers configuration}
\label{tconf2}
\end{figure}

TODO

\subsection{Case 3: Multiple Lasers and camera}

\begin{figure}[ht!]
\centering
\includegraphics[scale=0.7]{fig/4/test_configuration3.pdf}
\caption{Multiple lasers and camera configuration}
\label{tconf3}
\end{figure}

TODO

\subsection{Results and Comparison}

TODO